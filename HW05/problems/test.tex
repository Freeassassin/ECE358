Let's break down each part of the question:

**Part a)**

Consider the nodes of \(T_{opt}\) in the order of clockwise traversal A. We split the traversal into parts \(C_1, C_2, ..., C_l\) (l is the number of parts in A). The sum of the weights of these parts, \(\sum^{l}_{i=1} w(C_i)\), represents the total weight of the minimum Steiner tree \(T_{opt}\).

**Part b)**

Consider the largest part (in terms of edge weight) in the set of \(C_i\) denoted as \(C_{max}\). The weight of \(C_{max}\) represents the maximum weight among the parts.

**Part c)**

Consider \(P = \frac{A}{C_{max}}\). We want to derive a lower bound for the quantity \(P \leq 2(1 - 1/l) w(T_{opt})\) using the answers to the questions a-b.

Since \(A\) is the total weight of the minimum Steiner tree \(T_{opt}\), and \(C_{max}\) is the weight of the largest part, we can express \(A\) as \(A = \sum^{l}_{i=1} w(C_i) \geq l \cdot w(C_{max})\). Therefore, \(w(C_{max}) \leq \frac{A}{l}\).

Now, we have \(P = \frac{A}{C_{max}} \geq \frac{A}{\frac{A}{l}} = l\). Rearranging, we get \(l \leq P\).

Using the result from Part b, \(w(C_{max}) \leq \frac{A}{l}\), we have \(P \cdot w(C_{max}) \leq A\). Substituting \(P \leq l\) into this inequality, we get \(l \cdot w(C_{max}) \leq A\).

Now, we can use the fact that \(A \geq l \cdot w(C_{max})\) to derive a lower bound for \(P\):

\[P = \frac{A}{C_{max}} \leq \frac{A}{\frac{A}{l}} = l\]

\[P \leq l\]

**Part d)**

\(P\) contains each edge of \(T_{opt}\) at least once, and maybe twice. The weight of \(P\) is the sum of the weights of the edges in \(T_{opt}\). Let \(W_{opt}\) be the weight of \(T_{opt}\). Since each edge is counted at least once in \(P\), we have \(W_{opt} \leq w(P)\).

**Part e)**

In the algorithm, \(G_2\) is a minimum spanning tree of \(G_1\), and \(G_4\) is a minimum spanning tree of the graph \(G_3\) obtained by substituting edges of \(G_2\) with corresponding shortest paths in \(G\). 

Consider the total weight of \(G_3\), denoted as \(W_{G_3}\). Since \(G_4\) is a minimum spanning tree of \(G_3\), \(W_{G_3} \geq W_{G_4}\).

Now, \(G_3\) contains all the edges of \(G\), so \(W_{G_3} \geq W_{opt}\). Therefore, \(W_{G_4} \geq W_{opt}\).

**Part f)**

Combining the results from parts a-e, we have:

\[P \leq l \leq \frac{A}{C_{max}} \leq \frac{2A}{l}\]

\[P \leq \frac{2A}{l}\]

Since \(W_{opt} \leq W_{G_4}\) (from part e) and \(W_{G_4} \leq w(P)\) (from part d), we can combine these results:

\[W_{opt} \leq W_{G_4} \leq w(P)\]

\[W_{opt} \leq w(P)\]

Finally, substituting \(P \leq \frac{2A}{l}\) into the inequality, we get:

\[W_{opt} \leq w(P) \leq \frac{2A}{l}\]

\[W_{opt} \leq \frac{2A}{l}\]

This completes the proof:

\[w(T_{approx}) \leq 2(1 − \frac{1}{l}) w(T_{opt})\]