\section{}
\subsection*{Part a}
\textbf{Expression for Losses:}

To minimize the waiter's losses, we want to minimize the impatience-weighted waiting time for all clients. Let's define the total loss $L$ as the sum of losses for all clients:

\[L = t_1 * o_1 + t_2 * o_2 + t_3 * o_3 + \dots + t_n * o_n  \]

where:

$t_i$ is the impatience of the $i$-th client.

$o_i$ is the time taken to take the order from the $i$-th client.

The goal is to minimize $L$.

\textbf{Greedy Algorithm:}

1. Sort the clients based on their impatience, in non-decreasing order (i.e., in ascending order of $t_i$).

2. Take the orders from the clients in this sorted order.

This greedy algorithm sorts the clients by impatience and serves the least impatient clients first. The runtime of this algorithm is dominated by the sorting step, which has a time complexity of $O(n log n)$, assuming a standard sorting algorithm like quicksort is used.

\subsection*{Part b}
After serving the first client with the highest impatience ($t_1$), we are left with a subproblem of serving the remaining $n-1$ clients. This subproblem is identical to the original problem, but with one less client to serve. Therefore, by serving the first client greedily, we reduce the problem to a smaller subproblem with $n-1$ clients.

\subsection*{Part c}
To prove the greedy choice property, you need to show that there is an optimal solution that agrees with the first greedy choice your algorithm makes.

We can assume an optimal solution where the first client served has impatience $t_k$, and let the clients be reordered such that $t_1 \geq  t_2 \geq  \dots \geq  t_k \geq  \dots \geq  t_n$. If $k > 1$, there must be a client with impatience $t_j$ (where $j < k$) who is served before client $k$ in the optimal solution.

Now, consider swapping the positions of clients j and k in the optimal solution. This swap can only reduce the total loss or keep it the same because:

\[L_{j_new} = t_j * (w_j + w_k) (new wait time for client j)\]
\[L_{k_new} = t_k * w_k (new wait time for client k)\]

Since $t_j \leq  t_k$, and $w_j + w_k \geq  w_k$, it follows that $L_{j_new} \leq  L_{k_new}$.

Therefore, by swapping clients $j$ and $k$ in the optimal solution, we obtain a solution with a smaller or equal total loss. This contradicts the optimality of the original solution, showing that the first greedy choice (serving the client with the highest impatience) is also optimal.



\subsection*{Part d}
The problem exhibits a good substructure because the optimal solution to the original problem can be constructed from the optimal solutions to its smaller subproblems. By serving the least impatient client first and solving the subproblem of the remaining clients, you are creating an optimal solution for the original problem. This is because the greedy algorithm minimizes the losses for the current client, and by recursively applying it, you minimize the losses for the entire set of clients.

In summary, the greedy algorithm sorts clients by impatience and serves them in ascending order of impatience, which minimizes the losses. The problem exhibits the greedy choice property and optimal substructure, making this approach effective and optimal. The runtime of the algorithm is dominated by the sorting step, which is typically $O(n log n)$ for $n$ clients.




