\section{}

\subsection*{Part a)}

There are $N$ days, the list of deadlines on each day is: $d=[d_1,\ d_2,\ \ldots,\ d_N]$, with $d[i]$ deadlines on day $i$.

The given situation is a zero sum game, meaning that any deadlines Xun handles directly takes away from deadlines that Yuntao handles and vice versa, so both students will be attempting to maximize their score which will inherently minimize the other student's score. This type of problem is known as a maximin or minimax problem. In a zero sum game, a maximin and a minimax problem are identical. The minimax problem is formally stated as:
\[v_x=\max_{a_x}{\min_{a_y}{v_x(a_x,a_y)}}\]
In this equation, $v_x$ represents the value function of Xun, $a_x$ is the decision made by Xun, and $a_y$ is the decision made by Yuntao. In essence, Xun wants to make a decision that will ensure the maximal value function for herself after Yuntao makes whatever decision minimizes the value for Xun. This becomes a recursion problem as after Xun and Yuntao make one decision each, they are left with the same minimax problem but with a reduced list of deadlines. For example, if Xun and Yuntao both decided to do 1 day of deadlines each, they would be left with a minimax problem on $d=[d_3,\ d_4,\ \ldots,\ d_N]$

To properly evaluate the value function for Xun at any point, the 3D decisions Xun, is able to take must be evaluated, and for each of these decisions, all of Yuntao's decisions must be evaluated, at this point another value function for Xun must be evaluated. Another way to interpret this problem is to recognize that Yuntao's value function follows the same process as Xun's value function, so for the sake of the evaluating runtime, you can view the recursive cycle as: for every value function, evaluate a value function at each of the 3D decisions that can be made. The runtime of making any decision is constant, because of this, the recurrence can be stated as:
\[T\left(n\right)=\ \sum_{i=1}^{3D}{T(n-i)}+\theta(1)\]

\subsection*{Part b)}

\textbf{Efficient Solution}

To improve the efficiency of this solution, the value function can be computed for every day in the list and their potential $D$ values, working backwards from day $N$ towards day 1 and remembered. For example, the value function at day $N$ will be computed first, which is just equal to $d_N$. At days $N-4$ and earlier, the value of the value function will be dependent on $D$, so the value function will be calculated based on all potential $D$ values that yield different values, e.g. day $N-4$ will have the value calculated for $D=1$ and $D=2$, but not any other values since the $D \leq 3$ case is identical to the $D=2$ case.

When working backwards, recursion isn't needed as the value function for any days greater than the current day can simply be looked up from memory rather than recomputed. For days $N-2$, $N-1$, and $N$, the value function only needs to be computed once each, for days $N-5$, $N-4$, and $N-3$, the value function must be computed twice, once for each relevant value of $D$. This pattern repeats, for days $N-3k+1$,$ N-3k+2$, and $N-3k+3$, the value function must be computed $k$ times. Computing the value function once takes $O(k)$ time where $k$ is the same as previous because it must check 3D decisions and $k=D$, so the run time is correlated to $k$ multiplied by how many times the value function must be called. The run time can be expressed as: $\sum_{i=1}^{N}{\left\lceil\frac{i}{3}\right\rceil\ast i}$, this function is effectively adding the number of value function calls for every day which is $\left\lceil\frac{i}{3}\right\rceil$ multiplied by the number of possible decisions which will always be less than or equal to $i$. Notice that this function is strictly less than $\sum_{i=1}^{N}i^2$ which is equal to $\frac{N(N+1)(N+2)}{6}$, this means that the function run time is bound by $O(n^3)$. For the space complexity, one value must be saved for every time the value function called, which is $\sum_{i=1}^{N}\left\lceil\frac{i}{3}\right\rceil$, note that this function is strictly less than $\sum_{i=1}^{N}i$ which is equal to $\frac{N(N+1)}{2}$, this means that that the function space complexity is $O(n^2)$.

Therefore, the running time complexity is $O(n^3)$ and the space complexity is $O(n^2)$.

\textbf{Inefficient Solution}

The inefficient solution is the solution noted in part a) of this question, in which the program works from the front of the list to the end with no memorization. The space complexity is very simply $O(1)$ because no values are memorized.
To compute the runtime, it must be analyzed how often subproblems are called and how fast they are to solve. Whenever a value function is called, it must call a number of value functions equal to the number of decisions, each of these sub value functions will repeat this process. The absolute upper bound on the number of decisions that can be taken on a day is N as there simply cannot be more than N days to choose from. The next step is to analyze the worst case amount of calls that can be experienced. Because there are N days to step through, the worst case here is a depth of N calls. Because each call must consider N decisions and each path can be N calls of the value function deep, this function must be bounded by $O(N^N)$.
Therefore, the inefficient solution has runtime complexity of $O(N^N)$ and space complexity of $O(1)$.
